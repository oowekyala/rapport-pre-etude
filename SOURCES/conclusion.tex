\chapter*{Conclusion}
\addcontentsline{toc}{chapter}{Conclusion}

\section*{Retour d'expérience}

\subsection*{Choix du sujet, problèmes rencontrés et bilan}

Premièrement parti sur le sujet du système carcéral, le manque de réelle problématique, d’informations et simplement de motivation face au sujet nous on amené à revoir notre problématique de départ. Les recherches sur notre premier sujet nous avaient amené à aborder le cannabis, nous en avons finalement fait notre sujet d’étude. 

Ce thème s’est révélé fort intéressant, il nous a permis d’en apprendre beaucoup sur lui et de nous rendre compte de tout ce qu’il pouvait cacher. Mais ce sont notamment tous ses mystères qui ont rendu la réalisation de ce rapport compliquée. La recherche d’information sur le thème est difficile à cause des enjeux, des théories du complot, des blocages et des intérêts portés sur le cannabis. Beaucoup d’informations ne sont pas encore prouvées et il plane comme l’impression que rien n’est fait pour cela, ou bien encore qu’il est fait en sorte que cela reste ainsi. De plus c’est un sujet où il est difficile de trouver et surtout d’accepter des informations allant contre son avis de base, notre groupe n’est globalement pas contre la légalisation du cannabis, il est même plutôt en sa faveur, il nous fut donc difficile de rester subjectif. 

Au final nous en tirons le sentiment que la société se désinhibe peu à peu face au sujet du cannabis, des discussions et débats éclosent chaque jour, des évolutions sont observées. Il ne sera pas surprenant de voir le sujet du cannabis prendre de plus en plus de place dans les débats politiques de notre pays. Ils commencent déjà aujourd’hui même à faire parler de lui à l’approche des présidentielles de 2017, le 8 janvier 2017, 150 personnalités de Marseille demandent la "légalisation contrôlée" du cannabis.



\subsection*{Méthodes de travail}
Avant de rédiger notre monographie, nous avons collecté autant d’informations que possible autour du sujet du cannabis. La première difficulté fut de trier les informations à la fois utiles et fiables. Quand nous avons eu suffisamment d’informations pour avoir une problématique claire, chacun a choisi une partie puis l’a développé. Une fois que les trois parties furent rédigées, nous avons mis en commun notre travail pour être plus cohérents et ainsi éviter les répétitions. Enfin, nous avons fait la mise en page. Pour rendre le travail de chacun plus clair, nous aurions dû plus rapidement choisir le responsable de chaque partie afin de pouvoir mieux cibler nos recherches. De plus à ne vouloir se fermer aucune porte, nous avons surement dépensé trop d’énergie et de temps dans des recherches inutiles (pour le PSH).


\subsection*{Module PSH}

Nous avons apprécié, lors de ce module, d’être libres de choisir notre sujet ainsi que notre méthode de travail.

Cependant plusieurs demandes lors des séances dédiées aux modules (demande de plans, de listes de questions sur le sujet etc) nous ont détourné de nos recherches sur le sujet et ont été jugés assez peu utiles pour le bon fonctionnement de notre travail.

La découverte de nouveaux outils tel que Zotero fut cependant appréciée.







