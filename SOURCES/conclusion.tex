\chapter*{Conclusion}
\addcontentsline{toc}{chapter}{Conclusion}

Dans le cadre de notre projet de quatrième année en informatique à l'INSA, nous devons mettre en place une étude de trace sur le business game E-Yaka. Le but de cette opération est de collecter des données sur le comportement des apprenants, ce qui permettra d'améliorer l'application, aussi bien sur le plan ergonomique que pédagogique, et de transmettre des informations utiles à l'encadrant de la partie (par exemple, le fait qu'un apprenant ne se connecte plus et a visiblement abandonné la formation).


Nous avons d'abord étudié le jeu E-Yaka, qui met en place une simulation de gestion d'entreprise assez poussée. L'application est immersive : les apprenants jouent le rôle de véritables employés de l'entreprise, avec la possibilité de démarcher personnellement des clients. Cet aspect est sans doute l'un des points forts de l'application, car il permet un travail d'équipe tout en offrant aux apprenants la possibilité de se démarquer, ce qui aura un impact positif sur leur motivation.


Nous nous sommes ensuite renseignés sur les Learning Analytics, une discipline assez récente qui consiste à étudier des données en rapport avec une formation, pour comprendre le processus d'apprentissage et l'améliorer. Malgré des problèmes éthiques qui peuvent surgir dans certaines circonstances, les Learning Analytics ont un fort potentiel pour le futur de l'éducation, notamment pour créer des formations qui s'adaptent aux besoins et au fonctionnement des différents profils d'élèves.


Pour la mise en pratique du projet, nous avons fait des recherches sur les outils existants pour mettre en place des logs. Le plus prometteur semble être Experience API, une norme moderne conçue spécifiquement pour stocker des données en rapport avec l'apprentissage. De plus, une bibliothèque pour simplifier l'utilisation d'Experience API existe en PHP, le langage dans lequel E-Yaka est programmé. Il s'agit donc probablement de la technologie que nous allons retenir.


La prochaine étape du projet consiste à établir une spécification fonctionnelle pour le travail à faire. Pour cela, nous devrons décider précisément à quelles questions nous voulons être en mesure de répondre avec l'étude de traces, ce qui nous permettra de préciser le cahier des charges, et par la suite, de choisir des solutions d'implémentation de façon définitive.