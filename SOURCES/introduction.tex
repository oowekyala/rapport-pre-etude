\chapter*{Introduction}
\addcontentsline{toc}{chapter}{Introduction \guy{Jason}}
\pagenumbering{arabic}

	Le projet sur lequel nous devons travailler cette année concerne un business game nommé E-Yaka, un jeu de simulation d’entreprise. Ce projet est encadré par Fanny Gourret et Yann Ricquebourg en collaboration avec le SupTICE\footnote{Service Universitaire de Pédagogie et des TICE} de l’Université de Rennes-1 ainsi que le laboratoire LOUSTIC\footnote{Laboratoire d'Observation des Usages des Technologies de l'Information et de la Communication}. L’application permet à la fois de se familiariser avec le domaine de l'entrepreneuriat mais également de mettre en pratique les connaissances acquises dans ce domaine avant de les mettre en œuvre dans le cadre d’un véritable projet professionnel.

	Le jeu E-Yaka a été développé par le SupTICE et présente d’ores et déjà un grand nombre de fonctionnalités. Nous arrivons donc avec un projet qui a été amélioré au cours des années, notamment l’année dernière par des étudiants de 4ème année de l’INSA. Leurs travaux consistaient à mettre en valeur les nombreuses données de l’application par l'intermédiaire d’une carte interactive.

	Le jeu étant presque complet, il est nécessaire à présent d’y ajouter une couche qui se veut pédagogique. En effet, il serait souhaitable de pouvoir étudier le comportement des joueurs afin de savoir ce qu’ils retiennent réellement lorsqu’ils jouent à E-Yaka et d’améliorer l’ergonomie de l’application en conséquence de cette analyse.

	Ce domaine d’étude porte un nom : les \emph{Learning Analytics} ou analyse de l’apprentissage. L’objectif est de capturer le comportement des apprenants afin de leur proposer des retours précis. Les \emph{learning analytics} doivent permettre d’améliorer de façon significative l’efficacité des dispositifs d’apprentissage que sont les \emph{Business Game} tels que E-Yaka par exemple.

	Au moyen d’une analyse de traces, il nous est demandé de tracer le comportement des joueurs dans l’application. Le but est de repérer les actions des joueurs sur le jeu et de dégager les comportements anormaux puis d’y remédier. 
Par exemple, les joueurs doivent répondre à un questionnaire à propos d’informations disponibles dans l’application. Une des questions demande à l'apprenant le chiffre d’affaire de son entreprise, notre objectif est alors de retracer le cheminement lui ayant permis d’accéder à cette information. Si le joueur peine à la trouver, c’est probablement que l’information est mal placée et qu’il faut par conséquent modifier l’interface. 

	À la fin de notre projet nous devrions être en mesure de répondre à des questions telles que :
	\begin{itemize}
		\item Est-ce que les joueurs qui consultent le résultat du tour obtiennent de meilleurs résultats ?
		\item Est-ce que les joueurs qui utilisent les ressources vidéos obtiennent de meilleurs résultats ?
		\item Qu’est-ce qu’une bonne stratégie sur E-Yaka ?
	\end{itemize}

	Ce projet devra donc permettre d’analyser le comportement des joueurs afin de l’améliorer en conséquence ainsi que de proposer des moyens pour mettre en forme les données récoltées. Enfin, nos travaux pourraient également être intégrés sur la plateforme OpenClassrooms au travers d’un MOOC.
