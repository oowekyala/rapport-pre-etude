\chapter*{Introduction}
\addcontentsline{toc}{chapter}{Conclusion}
\pagenumbering{arabic}


La France doit-elle changer sa position par rapport au cannabis ? Le gouvernement réprime fermement la production, distibution et consommation du cannabis. Cette politique a un impact visible sur la société française : en plus de coûter très cher au contribuable, des problèmes sociaux tels que la surpopulation carcérale ou la criminalité sont des conséquences probables de la prohibition. Nous nous posons ici des questions sur la relation entre ces problèmes et la position répressive du gouvernement. Nous cherchons également à savoir si le pays pourrait profiter d'un changement de politique.

Beaucoup de pays ont déjà adopté un cadre plus libéral à l'égard du cannabis. Nous prenons comme exemples l'Uruguay, le Colorado et les Pays-Bas, tous trois des pionniers dans cette législation. Nous prenons également comme point de comparaison la prohibition de l'alcool, effective aux États-Unis pendant les années 20. La situation à l'époque ressemble en effet beaucoup à la situation actuelle : comme l'alcool, le cannabis pourrait induire une dépendance et entraîner des problèmes de santé (« pourrait » puisque, dans le cas du cannabis, ces points sont disputés parmi la communauté scientifique). De plus, la prohibition est dnas les deux cas le terreau de réseaux criminels d'envergure, qui organisent le trafic.

Il apparaît que la légalisation de l'alcool dans les années 30 et du cannabis au Colorado en 2014 ont grandement réduit la criminalité. De plus, les États peuvent profiter financièrement de ce nouveau marché légal en percevant des taxes et en régulant les prix. Cependant plusieurs régimes peuvent être distingués, ce que nous explorons dans la partie 2.