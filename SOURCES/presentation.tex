%\chapter*{Présentation synthétique}

\begin{abstract}
    \paragraph{Membres du groupe} Nous sommes un groupe de quatre étudiants du département Informatique, en troisième année : Lucas Clément, Jordan Le Bongoat, Clément Fournier et Élise Pottier.

	\thispagestyle{empty}

    \paragraph{Choix du sujet} Dans le cadre de ce projet de sciences humaines, nous nous sommes intéressés à un sujet particulièrement d’actualité dans ce contexte d’élections présidentielles: \textit{la position de l'État français sur le cannabis}.
    
    Au départ, nous étions surtout intéressés par le système carcéral français. Le sujet étant vaste, nous avons voulu préciser un peu plus notre champ de recherche. Nous nous sommes alors aperçus qu’un débat très actif ces dernières années est celui sur la légalisation du cannabis. Cependant, le débat ayant un aspect politique important, de nombreux éléments trouvés sur internet ne sont pas très objectifs. Comme nous vous l’expliquerons par la suite, certains acteurs de notre société ont beaucoup d’intérêts à ce que le cannabis ne soit pas légalisé. 
    
    Nous verrons donc dans un premier temps si la prohibition (plus particulièrement, celle du cannabis) peut favoriser la criminalité. Puis, nous vous présenterons les différences entre légalisation, décriminalisation et les conséquences que cela peut avoir. Enfin, nous étudierons comme notre société pourrait évoluer si on légalisait le cannabis. Tout cela en nous appuyant sur des cas concrets, des situations existant déjà dans d’autres pays du monde, notamment l’Uruguay et les États-Unis.
\end{abstract}