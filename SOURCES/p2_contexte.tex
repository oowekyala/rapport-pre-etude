\chapter{Description du projet}

    Intro: Eyaka se diversifie vers des Moocs, essaie d'appliquer des stratégies de learning analytics, puis explications


    Plan provisoire

    \section{Contexte  }
        \subsection{MOOC  \guy{Raph}}
        \subsection{Learning analytics \guy{Tom, Quentin}}
   
        \subsubsection{Définition et origine}
            
            Les Learning Analytics, aussi appelés \enquote{analyse de l’apprentissage} en français, décrivent l’étude des méthodes d’apprentissage et des résultats des étudiants dans un but pédagogique. Ils s’intéressent à  la mesure, la collecte, l’analyse et la présentation de rapports basés sur les données des apprenants en contexte d’apprentissage dans le but de comprendre et d’optimiser l’apprentissage et le contexte. La tendance des Learning Analytics vient des pays anglo-saxons, en particulier les Etats-Unis, la Grande Bretagne et l’Australie, où de nombreuses universités ont déjà mis en place une méthode d’apprentissage bénéficiant de l’apport de  l’analyse de l’apprentissage.

            Ceux-ci sont nés de plusieurs facteurs : la croissance récente et rapide de l’éducation en ligne, un intérêt grandissant pour les Big Data, notamment concernant l’informatique décisionnelle (outils informatiques destinés aux dirigeants d’entreprises, leur fournissant une aide à la décision grâce à la collecte et l’analyse de données) qui entraîne une volonté de transposer ces techniques jusqu’alors réservées au domaine de l’entreprise dans le domaine de l’éducation, ainsi qu’une demande importante de la part des enseignants qui souhaitaient pouvoir suivre de plus près l’avancement de l’apprentissage de leurs étudiants et offrir une expérience d’apprentissage personnalisée à chacun d’entre eux. 
       
            Le premier programme d’apprentissage exploitant les Learning Analytics a été lancé à la rentrée 2015 à l’université de Columbia par Ryan S. Baker, un professeur d’informatique à l’université de Pennsylvanie.
       
       \subsubsection{Objectifs et fonctionnement}
    
           À l’origine, l’objectif des Learning Analytics était de prédire le succès ou l’échec de l’apprentissage d’un étudiant en fonction de son origine et de sa base de connaissances. A présent, l’objectif premier des Learning Analytics est de comprendre et d’optimiser l’apprentissage et les environnements dans lesquels il se produit. Cependant, il ne s’agit pas du seul but. En effet, une autre application des Learning Analytics est l’« Early Warning System », c’est-à-dire la capacité à repérer un étudiant en cours de décrochage pour permettre à l’enseignant ou l’animateur d’intervenir. Pour cela, on peut par exemple se baser sur les résultats de l’étudiant à des questionnaires ou plus simplement sur la fréquence de ses connexions au site d’apprentissage. On peut aussi utiliser les Learning Analytics pour comparer les performances de plusieurs approches pédagogiques, par exemple en étudiant ce qui est le plus efficace entre un cours sous forme de texte et un cours sous forme de vidéos. Enfin, l’analyse de l’apprentissage permet d’identifier les points du cours qui posent problème, en repérant les parties du cours sur lesquelles les étudiants reviennent le plus souvent et en examinant les résultats obtenus sur les différentes questions d’un quiz.
       
           Sur le long terme, l’objectif des Learning Analytics est de permettre la mise en place de ce qu’on appelle « l’Adaptive Learning », c’est-à-dire d’offrir à chaque étudiant un parcours éducatif personnalisé et adapté à ses qualités. Cela dit, en plus d’une analyse de l’apprentissage efficace, l’Adaptive Learning demande du temps, de l’argent et une grande compétence technique. Plusieurs projets de recherche vont cependant dans ce sens, avec comme objectif d’aller toujours plus loin dans l’analyse du comportement de l’apprenant. Partant du constat que les manuels sont figés et ne s’adaptent pas à chaque apprenant, le projet HyperMind a par exemple été lancé. Celui-ci a pour but de créer un manuel scolaire s’adaptant au comportement de l’apprenant au moment de la lecture. En effet, grâce à une technologie d'eye-tracking et une caméra infrarouge, il est possible de mesurer les états cognitifs des apprenants et de faire évoluer le contenu du manuel en fonction de ceux-ci. Ainsi si l’apprenant passe plus de temps à relire un certain passage du manuel, cela implique qu’il lui pose certaines difficultés. Le contenu du manuel peut alors évoluer afin de fournir plus d’informations sur cette partie du texte visiblement incomprise par l’apprenant. Cela permet de proposer une expérience personnalisée pour chaque apprenant et ainsi optimiser l’apprentissage.
       
           À l’heure actuelle, les Learning Analytics sont très présents dans le domaine de l’éducation, que ce soit au sein des cours dispensés à l’école ou en dehors. Des outils comme EducLever, Kwiw ou encore Maxicours utilisent ce genre de techniques pour évaluer l’efficacité de l’apprentissage des élèves. Concernant le lycée et les études supérieures, les MOOCs se développent de manière assez fulgurante. Ces MOOCs utilisent les Learning Analytics afin de récolter des données sur les apprenants. Par exemple, pour savoir si un étudiant a réellement suivi un cours jusqu’au bout et sinon à quel moment il a abandonné et pour quelle raison, afin de pouvoir améliorer leur méthode pédagogique.
        
            Une plateforme ouverte a été créée et mise à disposition par le consortium Apereo afin de faciliter l’intégration des Learning Analytics dans le monde. Cette plateforme, intitulée « Apereo Learning Analytics Initiative » (ALAI) a été mise en place dans plusieurs universités américaines. Cette plateforme commence à être mise en place en France par le consortium national ESUP-Portail grâce au financement du Ministère de l’Éducation Nationale, de l’Enseignement Supérieur et de la Recherche (le M.E.N.E.S.R).
       
            Des tests vont ainsi être réalisés à l’université de Lorraine. Celle-ci possède une plateforme de plus de 11000 cours en ligne.
            Les objectifs dans le cadre de ce projet sont :
            \begin{itemize}
                \item Récupérer les traces laissées par les étudiants sur cette plateforme et de stocker ces données dans une base de données NoSQL.
                \item Relier ces données à d’autres informations concernant les étudiants comme leurs notes, leurs diplômes…
                \item Traiter ces informations grâce à un algorithme prédictif basé sur des technologies Big-Data fourni par la plateforme.
            \end{itemize}


           Tout ceci aura pour but de détecter les étudiants ayant un profil à risque, ce qui permettra aux enseignants d’optimiser leur apprentissage en proposant notamment des ressources spécifiques ou encore des entretiens particuliers.
        
           On peut donc distinguer trois étapes de traitement des données qui appartiennent au domaine des Learning Analytics : la récupération des données, leur analyse, et leur mise en forme. Ces trois étapes nous concerneront dans le cadre de ce projet puisque l’objectif est de présenter nos résultats de manière efficace et ergonomique à l’animateur du jeu. Il nous faudra donc d’abord récolter les traces des utilisateurs, puis analyser ces données afin d’en tirer des informations intéressantes, et enfin mettre nos résultats en forme pour en faciliter la lecture.
      
        \subsubsection{Débats sur les Learning Analytics}
    
            Il est également important de noter que les Learning Analytics provoquent quelques débats d’ordre éthique sur les données personnelles puisque ceux-ci nécessitent « d’espionner » l’apprenant dans le cadre de son travail personnel et de suivre de très près ses performances. Parfois, on utilise aussi des informations personnelles comme l’âge, le sexe, l’origine socio-démographique et le niveau  d‘éducation des étudiants. Ces données pourraient par exemples s’avérer utiles à beaucoup d’employeurs pour réaliser leur recrutement, ce qui pourrait constituer une forme de discrimination. De plus, les plateformes collectant ces données pourraient être tentées de les revendre à des tiers à des fins marketing. De même, les modèles prédictifs utilisés lors de l’analyse des données posent problème car s’ils permettent de détecter les élèves ayant un comportement dit “à risque” dès leur plus jeune âge, cela pourrait les enfermer dans des sortes de “bulles d’échec”. Cependant, l’article 10 de la loi Informatique et Libertés stipule qu’aucune “décision produisant des effets juridiques à l’égard d’une personne ne peut être prise sur le seul fondement d’un traitement automatisé de données destiné à définir le profil de l’intéressé où à évaluer certains aspects de sa personnalité”. De plus, en France, la Commission Nationale de l’Informatique et des Libertés (CNIL) a appelé le 23 mai 2017 à un “encadrement des services numériques dans l’éducation”, en incitant les fournisseurs de logiciels éducatifs et les fournisseurs d’accès internet à adopter une Charte de confiance engageant ces fournisseurs à respecter les droits des personnes. Cette charte se traduit notamment par “un encadrement juridique contraignant tant en ce qui concerne la non utilisation des données scolaires à des fins commerciales, l’hébergement de ces données en France ou en Europe ou encore l’obligation de prendre des mesures de sécurité conformes aux normes en vigueur”.
        
            D’autres controverses existent concernant les learning analytics. Par exemple, les learnings analytics peuvent être critiquées dans le sens où elles ne permettent d’analyser qu’une petite partie du travail de l’apprenant. En effet, l’apprentissage ne se fait pas uniquement grâce aux outils numériques mais aussi en classe, ce qui est difficilement mesurable. De même, l’analyse de l’apprentissage se traduit le plus souvent par un score ou l’obtention de diplômes, sans prendre réellement en compte des connaissances et habiletés acquises par l’apprenant.
        
            De mêmes, certaines personnes critiquent la capacité des enseignants à lire et analyser les tableaux de bord produits lors de la collecte des données et à mettre en relation les informations qu’ils fournissent avec les comportements des apprenants, car la plupart du temps ce sont eux qui sont chargés de traiter ces résultats et faire évoluer leur pédagogie, ce qui les rapproche de l’ingénierie pédagogique, chose pour laquelle ils ne sont pas nécessairement formés.
        
       


    \section{Cahier des charges  \guy{Arnaud pour rédaction}}

        Problématique

        \subsection{Collecte de traces}


        \subsection{Analyse des traces}