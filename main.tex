\documentclass[a4paper,12pt,twoside]{report}

\makeatletter

	\usepackage[french]{babel}
	\usepackage[utf8]{inputenc}
	\usepackage[T1]{fontenc}
	
	
	\usepackage%
		{xcolor}
	
	\usepackage{xkeyval}
	\usepackage{geometry}
	\usepackage{graphicx}
	\usepackage{lipsum}
	\usepackage{xspace}			
	\usepackage{lmodern} 
    \usepackage{titling}
	\usepackage{sectsty}
	
	\usepackage{csquotes}


	
	\definecolorset{HTML}{}{}{belux,900A40;underbelux,AA526A;antibelux,000000}	% 
	
	
	\allsectionsfont{\color{black}}					
	\partfont{\color{belux}}						
	\chapterfont{\color{belux}}						
	\sectionfont{\color{belux}}						
	\subsectionfont{\color{belux}}					
	\subsubsectionfont{\color{underbelux}}				
	



%---------------------------
%% GEOMETRY MANAGEMENT

	\def\definegeometry#1#2{									%%
		\define@key{modelgeometries}{#1}[]{\geometry{#2}}			%%
		\define@key{setgeometrykeys}{#1}[]{\newgeometry{#2}}		%%
	}


	%%--------------------- \setmodelgeometry

	\newcommand{\setmodelgeometry}[1]{
		\setkeys{modelgeometries}{#1}
	}


	%%--------------------- setgeometry environment

	\newenvironment{setgeometry}[1]
		{\setkeys{setgeometrykeys}{#1}}
		{\restoregeometry}


	
	

	%% GEOMETRIES LIBRARY
	%%-------------------

	\definegeometry{math}			%% MATH
		{					
			inner=90pt,		
			bottom=65pt,
			top=65pt,
			outer=70pt			
		}
	
	\definegeometry{cover}			%% COVER - used in front_management
	{					
		top=3.5cm,			
		bottom=0cm,			
		inner= 0.6cm,		
		outer=0.6cm			
	}
	
	\setmodelgeometry{math}
	
	
\makeatother


%------------------
%------------------


%% PACKAGES
%\usepackage{caption,subcaption}
%	\captionsetup{singlelinecheck=off}

\usepackage{hhline}
\usepackage{eurosym}

\usepackage{tgcursor}

%%
\usepackage{appendix}
	\makeatletter
	%	\@addtoreset{chapter}{part}
		\@addtoreset{@ppsaveapp}{part}
	\makeatother

\usepackage[hidelinks]{hyperref}


\addto\captionsfrench{\renewcommand{\chaptername}{Partie}}


%\ExecuteBibliographyOptions{
%datamodel=NE_PAS_TOUCHER/mynote.dbx,
%}
\usepackage[
	%backend=biber,
 	style=numeric,
  	%citestyle=authoryear,
  	sorting=none,
    datamodel=NE_PAS_TOUCHER/mynote.dbx,
]{biblatex}
%\DeclareDatamodelFields[type=field,datatype=literal]{mynote}
\usepackage{xpatch}
\xapptobibmacro{finentry}{\par\printfield{mynote}}{}{}


%% COVER PREAMBLE


\usepackage{latexsym} % anciens symboles latex
\usepackage{makeidx}
\usepackage{calc}
\usepackage{ifthen}
\usepackage{float}
\usepackage{epsfig}
\usepackage{vmargin}
\usepackage{textpos}
\usepackage{color}
\usepackage{letterspace}
\usepackage{fancybox}

\setlength{\hoffset}{0cm}
\setlength{\voffset}{0cm}
\setlength{\paperheight}{29.7cm}
\setlength{\paperwidth}{21cm}
\setlength{\textwidth}{14.5cm}
\setlength{\marginparwidth}{0cm}
\setlength{\headheight}{2\baselineskip}
\setlength{\textheight}{21cm}
\setlength{\evensidemargin}{0cm}
\setlength{\oddsidemargin}{1.5cm}


%% Listings
\usepackage{listings}
\lstset{								% general command to set parameter(s)
	basicstyle=\ttfamily\small, 				% print whole listing small 
	keywordstyle=\color{black}\bfseries\underbar, 	% underlined bold black keywords 
	identifierstyle=, 					% nothing happens 
%	commentstyle=\color{white}, 				% white comments 
	stringstyle=\ttfamily, 					% typewriter type for strings 
	showstringspaces=false,
	numbers=left,
	numberstyle=\tiny\ttfamily,
	language=Java,
	tabsize=4,
	literate={{é}{{\'e}}1 {è}{{\`e}}1 {ê}{{\^e}}1 {à}{{\`a}}1 {œ}{{\oe}}1},
	xleftmargin=0pt,
	keywordstyle=\bfseries\color{underbelux},
	literate=							%
			{<-}{{$\leftarrow$}}{1}		%
			{<--}{{$\longleftarrow$}}{1}%
			{->}{{$\rightarrow$}}{1}	%
			{<->}{{$\leftrightarrow$}}{3}	%
			{-->}{{$\longrightarrow$}}{3}	%
			{<-->}{{$\longleftrightarrow$}}{4}	%
			{-|>}{{$\nrightarrow$}}{1}	%
			{+*}{{$\times$}}{1} 		%
			{A/}{{$\forall$}}{1}		%
			{é}{{\'e}}{1}				%
			{è}{{\`e}}{1}				%
			{ê}{{\^e}}{1} 				%
			{à}{{\`a}}{1}				%
			{É}{{\'E}}{1}				%
			{’}{{'}}{1}
	}


\lstdefinelanguage{pseudocode}
{
% list of keywords
morekeywords={
si, sinon, finsi, fsi, alors,
tantque, while, tant, que,
for, do, fintantque,fin, pour,
fonction, renvoyer, appeler,
TAD, fonctions, utilise, preconditions, 
booleen, entier
},
sensitive=false, % keywords are not case-sensitive
morecomment=[l]{//}, % l is for line comment
morecomment=[s]{/*}{*/}, % s is for start and end delimiter
morestring=[b]", % defines that strings are enclosed in double quotes
keywordstyle=\bfseries\color{darkgray}
}

\lstdefinelanguage{json}
{
	string=[s]{"}{"},
    stringstyle=\color{underbelux},
	morecomment=[l]{:},
	commentstyle=\color{black},
}


\newcommand{\displaycode}[2][]{\begin{center}\lstinline[mathescape,#1]^#2^ \end{center}}
	

%% DEFINITIONS


\def\cad{c'est-à-dire}
\def\ie{\emph{i.e.}}
\def\guy#1{\textcolor{blue}{\bfseries #1}}











\input{NE_PAS_TOUCHER/couvresumesPFE.sty}



\pretitle{\Huge \begin{center} \bfseries \color{belux}}
\title{La France devrait-elle changer sa position sur le cannabis ?}
\date{\today}
\author{
	Lucas \textsc{Clément} \and
	Clément \textsc{Fournier} \and
	Jordan \textsc{Le Bongoat} \and
	Élise \textsc{Pottier}
}
\widowpenalty10000
\setlength{\parskip}{4pt}

  \bibliography{CannabisPSH}
\begin{document}
	
		{
			\pagestyle{empty}
			\pagestyle{empty}
\fontfamily{cmss}
\selectfont


\def\mate#1{\textbf{\color{white}\large #1}}

\newgeometry{top=0cm,bottom=0cm,left=0cm,right=0cm}

\fontfamily{cmss}
\hspace{-0.75cm}
\begin{minipage}{21cm}

%\vspace{-6.6cm}
\noindent \epsfig{figure=NE_PAS_TOUCHER/pfe-insa-page1.png,width=21cm}

\vspace{-27.6cm}
\hspace{12cm}\begin{minipage}{8cm}
\begin{flushright}
%{\color{white}{\Large\bfseries PFE}}

\mate{Jason \textsc{Barrier}}

\vspace{3mm}
\mate{Quentin \textsc{Bigot}}

\vspace{3mm}
\mate{Arnaud \textsc{Cornillon}}

\vspace{3mm}
\mate{Raphaël \textsc{Esteveny}}

\vspace{3mm}
\mate{Clément \textsc{Fournier}}

\vspace{3mm}
\mate{Jordan \textsc{Le Bongoat}}

\vspace{3mm}
\mate{Tom \textsc{Richardon}}

\vspace{3mm}
\mate{Pierre \textsc{Testart}}

\vspace{5mm}
{\color{white}Spécialité INFO}

\vspace{1mm}
%{\color{white} \today}
\end{flushright}
\end{minipage}
\end{minipage}

\vspace{12.7cm}
%\begin{textblock}{18}(-3,0)
\hspace{7cm}\begin{minipage}{11cm}
\noindent \Huge\bfseries \thetitle
	\baselineskip=20pt
    
 
{\Large \bfseries\color{gray} \today}

Yann Ricquebourg
Fanny Gouret
	
\end{minipage}
%\end{textblock}


%
\fontfamily{cmss}

\newpage
\restoregeometry

\fontfamily{cmr}

	%		\cleardoublepage		
		}

\newgeometry{top=2cm,bottom=2cm,inner=3cm,outer=2cm}
	{
		\pagestyle{empty}
	%	\maketitle
		\tableofcontents
        \thispagestyle{empty}
        %\chapter*{Présentation synthétique}

\begin{abstract}
    \paragraph{Membres du groupe} Nous sommes un groupe de quatre étudiants du département Informatique, en troisième année : Lucas Clément, Jordan Le Bongoat, Clément Fournier et Élise Pottier.

	\thispagestyle{empty}

    \paragraph{Choix du sujet} Dans le cadre de ce projet de sciences humaines, nous nous sommes intéressés à un sujet particulièrement d’actualité dans ce contexte d’élections présidentielles: \textit{la position de l'État français sur le cannabis}.
    
    Au départ, nous étions surtout intéressés par le système carcéral français. Le sujet étant vaste, nous avons voulu préciser un peu plus notre champ de recherche. Nous nous sommes alors aperçus qu’un débat très actif ces dernières années est celui sur la légalisation du cannabis. Cependant, le débat ayant un aspect politique important, de nombreux éléments trouvés sur internet ne sont pas très objectifs. Comme nous vous l’expliquerons par la suite, certains acteurs de notre société ont beaucoup d’intérêts à ce que le cannabis ne soit pas légalisé. 
    
    Nous verrons donc dans un premier temps si la prohibition (plus particulièrement, celle du cannabis) peut favoriser la criminalité. Puis, nous vous présenterons les différences entre légalisation, décriminalisation et les conséquences que cela peut avoir. Enfin, nous étudierons comme notre société pourrait évoluer si on légalisait le cannabis. Tout cela en nous appuyant sur des cas concrets, des situations existant déjà dans d’autres pays du monde, notamment l’Uruguay et les États-Unis.
\end{abstract}
		\listoffigures
        \begingroup
        \let\clearpage\relax
        \listoftables
        \endgroup
        \thispagestyle{empty}
		\cleardoublepage
	}
		
  			\chapter*{Introduction}
\addcontentsline{toc}{chapter}{Introduction \guy{Jason}}
\pagenumbering{arabic}


\addcontentsline{toc}{chapter}{\color{red}PROVISOIRE, n'hésitez pas à changer}

INTRO
			\chapter{L'interdiction favorise-t-elle la criminalité ?}

	\section{La prohibition: un terreau pour le développement des réseaux criminels}

\paragraph{Un débat relancé}

    La légalisation du cannabis est un sujet récurrent et régulièrement débattu, notamment par les politiques. Le 11 avril 2016, Jean-Marie Le Guen n’a pas fait exception en relançant le débat \cite{leguen16}. L’un de ses arguments étant que légaliser le cannabis baisserait la criminalité. En effet, en 2014, 10 \% des dossiers judiciaires concernaient la drogue et plus précisément 91 \% d’entre eux étaient liés au cannabis, le plus souvent à sa consommation. C’est pourquoi, une légalisation ou décriminalisation entraînerait une chute de la criminalité. Cependant, cette diminution ne serait absolument pas liée à une baisse de la consommation de cannabis. On ne peut donc pas vraiment considérer cela comme une solution au problème de la consommation du cannabis.

\paragraph{Le syndrome de la vitre cassée}

    En revanche, le syndrome de la vitre cassée implique une baisse de la criminalité.

    Le syndrome de la vitre cassée est une théorie parue et développée dans un ouvrage de Georges Kelling, un criminologue du New Jersey. Selon lui \cite{wikiVitreBrisee}, si on prend l’exemple d’un quartier résidentiel, le délabrement des bâtiments, notamment visible par une vitre brisée (d’où le nom), encouragerait à la criminalité et au dépravement. En effet, constater que les bâtiments ne sont pas entretenus, que les gens ne sont pas forcément très polis entre eux, donne un sentiment d’impunité et augmente par la même le nombre de petits délits. Ceux ci ne semblent plus très importants, au vue de l’environnement. Certaines personnes se sentent donc moins coupables, ont moins l’impression de faire quelque chose de mal et c’est ainsi que le taux de criminalisation augmente. Le fait de voir beaucoup de criminalité renforce le sentiment qu’on peut faire des choses illégales et augmente la criminalité. C’est ce cercle vicieux qu’on appelle le syndrome de la vitre cassée.

\begin{figure}\centering
\includegraphics[width=.8\textwidth]{images/vitrecassee.png}
\caption{Schéma d’explication du syndrome de la vitre cassée}
\end{figure}

    En reprenant l’exemple du cannabis, une légalisation permettrait effectivement de diminuer la criminalité, et pas seulement celle liée à la détention, la consommation ou la vente. En supprimant quasiment 140 000 infractions par an, la criminalité est moins exposée aux yeux de tous, notamment ceux des plus jeunes, donnant une impression plus reluisante de notre société et dissuadant ainsi des possibles criminels qui se sentiront plus « encadrés ».

    En admettant qu’une légalisation du cannabis permettrait de diminuer la criminalité, elle libèrerait par la même occasion une quantité non négligeable de places dans les prisons.
    

\section{Les prisons surpeuplées : une culture lucrative}

\paragraph{Pourquoi une privatisation des prisons ?}

    Suite à une surpopulation carcérale et de manière générale à des restrictions des budgets publics, il y a une trentaine d’années, un phénomène partant des États Unis mais se répandant très vite à travers le monde est apparu : les prisons privées. Cette privatisation prend deux formes : la privatisation totale et la gestion mixte.

    La privatisation totale \cite{dufresne10}, comme son nom l’indique est totale. L’entreprise privée assure donc toute la prise en charge de la prison, de sa construction à la surveillance des détenus. Ce type de prison n’est cependant pas présent en France, même si il reste utilisé aux États-Unis, au Royaume Unis ainsi qu’en Australie.

    La gestion mixe, aussi nommée Partenariat Public Privé, concerne les prisons dans lesquelles l’État a un réel contrôle. Il délègue cependant une partie de son pouvoir à certaines entreprises privées. Ces contrats s’étendent sur au moins 25 ans à partir de l’ouverture de la prison


\paragraph{La situation en France}

    La privatisation des prisons en France, si elle s’étend de plus en plus, a commencé de manière progressive avec la loi Chalandon en 1987, puis s’est renforcée avec la loi d’orientation et de programmation judiciaire de 2002. Ainsi, l’État doit obligatoirement conserver quelques tâches comme la surveillance des prisons ou la direction des prisons. Les autres tâches, ne nécessitant pas un pouvoir particulier, sont de plus en plus assurées par des entreprises privées. Au début, seules la conception et la réalisation des prisons étaient concernées par cette privatisation. L’État paye bien évidemment les entreprises pour les tâches qu’elles accomplissent, allant même jusqu’à leur verser un loyer proportionnel au coût de fabrication des bâtiments.

    Selon l’observatoire des multinationales, en 2016, 68 prisons étaient privées pour un total de 188 prisons. Cela représente environ 36 \% des prisons. De plus, plus de la moitié des prisonniers sont détenus dans des prisons privées. Les entreprises profitant de ce marché n’ont donc pas intérêt à ce qu’un nombre important de peines soient réduites, ou même supprimées comme cela serait le cas avec la légalisation du cannabis.

    Depuis 2008, nous somme entrés dans le type de privatisation évoqué précédemment : le partenariat public-privé. Selon l’Organisation Internationale des Prisons (OIP) \cite{knaebel16}, avec ces contrats, l'Etat doit verser en loyer environ 5,9 milliards d’euros par an. De plus, après s’être engagé à payer un loyer pour les 25 années à venir, l’État n’a aucun intérêt à réduire des peines ou à vider les prisons maintenant. 

    
\section{Le marché noir : une vitrine sur les drogues}

\paragraph{Un marché noir très présent}

    Actuellement, consommation, production et possession de cannabis ou autres drogues sont interdites (sauf pour raisons médicales). Comme toute interdiction, celle ci favorise le développement d’un marché noir et de toute une organisation criminelle comprenant gangs, dealer, etc. Cela représente une part de la criminalité

\paragraph{Le cannabis : une drogue d’introduction}

    Avec cette forte présence du marché noir, il est devenu facile de se procurer du cannabis. Si cette drogue appartient à la catégorie des drogues douces, ce n’est pas le cas de toutes les drogues vendues par le dealer auprès de qui les consommateurs se fournissent. C’est pourquoi, l’interdiction du cannabis peut d’une certaine façon donner un nouveau rôle à cette drogue : une drogue d’introduction.

    Quand le consommateur rentre en contact avec un dealer, celui-ci lui vend ce qu’il demande. Seulement à partir d’un certain temps, le consommateur peut, par curiosité ou pour d’autres raisons, s’intéresser à des drogues dures, plus dangereuses. De plus, on peut ici aussi appliquer le syndrome de la vitre cassée (expliqué dans la partie I.1). En effet, le consommateur étant déjà dans l’illégalité en achetant et consommant le cannabis, il ne verra pas de grande différence entre le cannabis et les autres drogues que peut lui fournir le dealer. À titre d’exemple sur une situation à peu près équivalente, aux États-Unis lors de l’interdiction de la consommation d’alcool, celle ci était devenue incontrôlable. 

%Transition :
On remarquera aussi que si une légalisation mettrait en échec le marché noir, ce n’est pas le cas d’une dépénalisation qui interdit toujours la vente.

   			\chapter{Différents scénarios}

En France, la position très répressive du gouvernement sur le cannabis coûte chaque année 568 millions d’euros au contribuable (source Terra Nova). Si la classe politique est peu encline à changer les choses, des pays comme les Pays-Bas, l’Uruguay ou le Colorado, pionniers dans cette législation, montrent  que ce n’est pas la seule solution.

Comment adapter la législation française pour en finir avec la répression ? Le groupe de réflexion Terra Nova, dans un rapport publié fin 2014, présente et chiffre trois régimes possibles, décrits dans les sections suivantes.

\section{Scénario 1 : Dépénalisation de l’usage}

    Le premier scénario concerne la dépénalisation de l’usage du cannabis. Par cela, on entend l’arrêt de la poursuite et de la punition des consommateurs ; cependant, la production et la vente de cannabis resterait illégale. Plusieurs pays ont déjà adopté cette politique, parmi lesquels nos voisins portugais et espagnols.

    \paragraph{Impact financier} Changer la politique de répression pour ne cibler que les producteurs et vendeurs réduirait significativement le coût de l’action policière : en France, les seules gardes à vue sur usage de cannabis constituaient 10\% du total des gardes à vue en 2013 (source OFDT). Le groupe de réflexion estime que les dépenses publiques de répression dans ce scénario se verraient amputées de 311 millions d’euros par an, ce qui représente une baisse de 55\%. En effets, les coûts d’opération de la police se verraient presque divisés par 7, tandis que les dépenses de justice, santé et prévention devraient rester à leur niveau actuel.
    
     \paragraph{Régulation du marché} Ce régime particulier ne permettrait pas de déraciner le marché noir, puisqu’il ne serait pas concurrencé par un marché légal. L’État n’aurait donc toujours aucun moyen de contrôler le prix du chanvre, qui agit directement sur la prévalence de l’usage. De la même façon, le Trésor public ne pourrait percevoir aucun revenu sur ce marché qu’il ne reconnaîtrait pas. 

    \paragraph{Coût psychologique en baisse} D’un point de vue sanitaire, supprimer les risques pénaux encourus par un consommateur en recherche du produit pourrait effectivement réduire le coût psychologique d’acquisition de l’herbe. Une telle baisse du coût tel qu’il est perçu par le consommateur ne pourrait qu’augmenter le nombre de consommateurs et la quantité de cannabis consommé. Cependant, il est à noter que le prix du marché (monétaire) resterait probablement stable, étant donné que ce scénario ne change rien à la situation des producteurs et distributeurs.

    \paragraph{Synthèse} Dans ce scénario, Terra Nova estime à 12\% l’augmentation du nombre d’usagers quotidien, et à 16\% l’augmentation du trafic en masse de produit vendu. Ces augmentations ne sont pas négligeables, et couplées au manque de contrôle de l'État sur le marché, donnent tort à cette solution sur le long terme.
    
    Les autres scénarios traitent le problème d'une autre manière, par la \textbf{légalisation}. La section suivante argumente en faveur de cette politique.





%Répression ou légalisation : la controverse sur le cannabis se résume à deux bataillons bien tranchés, où chacun campe sur ses positions. Cette polarisation n’est pas inintéressante ; elle est infiniment plus enrichissante qu’un quelconque unanimisme. Mais elle permet, à chaque parti du conflit d’idées, d’éviter soigneusement de procéder à l’analyse structurelle d’un phénomène dynamique. 
 
\section{Pourquoi légaliser le cannabis ?}
 
La légalisation du cannabis est le principe de lever tous les interdits pénaux sur la production, le commerce et l’usage du cannabis afin d’imposer un marché légal là où un marché noir est présent. L’usager du cannabis n’est plus dans ce système un délinquant ou un malade. Il est considéré comme un individu normal, sans doute affecté d’un vice, comme le jeu ou la boisson, mais les vices ne sont pas des crimes. La société les tolère tant qu’ils ne nuisent pas à l’ordre et à la moralité publique.

C’est pourquoi, dans un système de légalisation contrôlé, l’usage est interdit dans les lieux publiques, au volant, et en tout état de cause, réservé aux majeurs. Le cannabis n’étant pas une marchandise comme les autres, il est nécessaire de supprimer les pratiques qui constituent un encouragement à le produire, le vendre, ou à le consommer. Et toutes formes de promotion des ventes sont donc strictement interdites.

Ce sont ces principes qui ont été mis en œuvres à ce jour dans les quelques pays et Etats américains ayant légalisés le cannabis : En 2015, ils se comptent sur les doigts de la main : Uruguay, Colorado, Washington, Alaska, Californie , Oregon.

Ces initiatives sont encore trop fraîches pour en tirer des conclusions définitives, mais il ne serait pas étonnant que, par un effet domino, le succès de ces expérimentations ne serve d’exemple à notre pays. Certes, le principe en droit international demeure à ce jour celui d’une interdiction générale et absolue de ce produit, à l’exception de ces usages médicaux et scientifiques. Mais à la lumière de l’échec de la prohibition, les bénéfices attendus par la légalisation pourraient aider des pays à franchir le pas. Les gains indirects en termes de santé publique et de sécurité ne doivent pas être sous-estimés.

Par ailleurs, si l’on s’en tient à une lecture purement économique, il est probable que la légalisation soit une bonne affaire. La fiscalisation de ce marché et le développement d’une industrie du cannabis récréative constituerais un important gisement d’emplois, comme l’illustre l’exemple du Colorado.

La légalisation s’impose pour des raisons sanitaires, sécuritaires et économiques. Vouloir maintenir un interdit symbolique en se prévalant de la conviction qu’il est préférable de vivre sans drogue est tout à fait respectable. Il est en revanche irresponsable d’y voir une réponse juridique opératoire au défi que portent de manière aiguë la consommation et le trafic de cannabis.
 
\section{Scénario 2 : Légalisation de l’usage et la vente du cannabis dans le cadre d’un monopole public}
 
Ce scénario se penche sur la légalisation très fortement régulé par l’Etat, afin de placer le cannabis comme un bien marchand, tel que le tabac ou l’alcool, sous un monopole public.

Les principaux principes de la légalisation contrôlée par l’Etat visent à concilier le respect des libertés individuelles du consommateur de cannabis et les intérêts légitimes de la société. C’est la voie choisie par l’Uruguay : cette option permet à l’Etat de réguler le prix pour garantir une relative stabilité de la consommation. Une question se pose, à quel prix doit-on vendre le cannabis ?

\paragraph{Deux possibilités} On imagine deux scénarios possible dans ce cadre : garder le prix inchangé ou majorer la revente ---\,la possibilité de réduire le prix par rapport à celui du marché noir est directement exclue, pouvant être prise comme une incitation à la consommation.

\paragraph{À l'Uruguayenne} L’intérêt de garder le prix inchangé est d’assécher instantanément le marché noir, puisque le consommateur achètera au même prix, sans encourir de risque judiciaire, policier ou criminel. C’est (entre autres) pour cette raison que l’Uruguay est devenu, en 2013, le premier pays à contrôler intégralement la production, la vente et la consommation de cannabis. Mais cette solution engendrerait une hausse de la consommation totale de l’ordre de 185 tonnes (65\%) (53 tonnes de plus pour les usagers quotidiens et 132 tonnes de cannabis intégralement consommé par de nouveaux utilisateurs)  et une augmentation du nombre d’usagers quotidiens de l’ordre de 262 000 personnes (45\%).

\paragraph{À la Hollandaise} Le but de la légalisation, est de contrôler le nombre de consommateurs et non de le voir augmenter. C’est pourquoi une deuxième solution a été imaginée : majorer le prix de vente, « à la hollandaise ». Cette majoration a pour but de trouver une équivalence entre les prix pratiqués par le marché noir et le marché légal. En effet, la majoration correspondra à un coût que l’usager est prêt à supporter pour ne plus prendre de risque.

L’estimation (estimation forte) effectuée par Terra Nova avance une majoration de 40\% : 20\% pour le coût d’interpellation actuel (5\% pour les usagers et 15\% pour les revendeurs) et 20\% pour le risque lié au fait de côtoyer le marché illicite. Partant d’un prix moyen de 6\euro le gramme de cannabis sur le marché noir, le prix majoré serait alors de 8\euro 40 le gramme. Cette majoration permettrait donc d’avoir une augmentation du nombre d’usagers quotidien nulle, ainsi que du tonnage consommé. Cependant, le démantèlement des réseaux clandestins se fera alors plus progressivement que dans le cadre d’un prix inchangé puisque certains consommateurs seront encore attirés par des prix plus faibles, en dépit du risque encourus.

\paragraph{Impact sur les finances publiques} En se fondant sur les mêmes données et la même méthode que dans la partie précédente (Scénario 1 : Dépénaliser le cannabis), les frais de justices et de polices dûs à la légalisation seraient nuls. Seuls les frais de santé et de prévention seraient alors pris en compte, ce qui reviendrait à une dépense publique annuelle de l’ordre de 505 millions dans le cas d’un prix inchangé et de 523 millions dans le cadre d’une majoration des prix de 40\% (nombre de consommateurs moins importants que dans la proposition de prix inchangé, donc moins de frais sociaux et médicaux).
En terme de revenus publics sur les ventes, on doit prendre en compte seulement les ventes et non les parts de cannabis consommé provenant de l’auto-culture et de dons.\cite{durand_cannabis_2016}

\renewcommand{\arraystretch}{1.2}
\begin{table}\centering
\begin{tabular}{c|cccc}
&Total&Achat&Don&Auto-culture \\ \hline
En tonnes&277&208&37&32\\
En pourcents&100 \%&75.1 \% & 13.6 \% & 11.3 \% \\
\end{tabular}
\caption{Origine du cannabis (source : Ben Lakhdar 2009)}
\end{table}

 
Dans l’hypothèse 1 (prix non taxé, 6\euro) on obtient donc un chiffre d’affaires de 2.1 milliards et dans l’hypothèse que l'État taxe la vente à 80\% (comme celle du tabac), un revenu fiscal de 1.9 milliards.
Dans l’hypothèse 2 (prix taxé, 8,4\euro) on obtient un chiffre d’affaire de 1.7 milliards, qui génère 1,3 milliard à 80\% de taxes, en faisant l’hypothèse d’une éviction du marché noir. La \textbf{table \ref{tab:synthese2}} synthétise ces résultats.


 \begin{table}[p]\centering
 \hspace{-2cm}\begin{tabular}{p{3.8cm}|cccccc}
  &Statu quo& Légalisation		& Variation & Légalisation	& Variation \\
  &   		& prix de vente		&			& prix de vente		&			\\
  &  		& inchangé (6\euro)	&			& majoré (8\euro 40)& 			\\\hline \hline   
  Usagers quotidiens (en milliers)  & 550  	&812  	&+47.6 \%  	&550  	&0.0 \% \\\hline
  Volume trafic  (en tonnes)  		& 277  	&457  	&+65.0 \%  	&277  	&0.0 \% \\\hline
  Prix de vente (en euros)  		& 6  	&6  	&0.0 \% 	& 8.4  	&+40.0 \%\\\hline
  Coût d’acquisition (en euros) 	& 8.4	&6 		& -28.6 \%  &8.4  	&0.0\%\\\hline
  Dépenses publiques (en millions d’euros) 	& 568  	&65.8       &-88.4 \%  &44.6  &-92.1 \%\\ \hline
  Recettes publiques (en millions d’euros)	&0		& 1647	&& 1331 &\\\hline
 \end{tabular}
 \label{tab:synthese2}
 \caption{Tableau de synthèse du scénario 2 \cite{terraNova_rapport}}
\end{table} 
 
 
\section{Scénario 3 : Légaliser l’usage et la vente dans un cadre concurrentiel}
 
Ce scénario ressemble beaucoup au scénario 2, excepté une grosse différence : au lieu de s’inscrire dans le cadre d’un monopole public avec un prix fixé par l’Etat, le prix est décidé cette fois par le marché concurrentiel.
Ce scénario laisse imaginer que le prix du cannabis descendra rapidement à cause de la concurrence, ce qui entraînera une forte hausse de la consommation.
Pour une baisse de prix de vente de 10\%, Terra Nova estime que la consommation en tonnage se verra doubler (544 tonnes au lieu de 274tonnes) et le nombre de consommateur quotidien augmentera de 393 000, soit 71\%.
L’État collecterait 1.7milliards d’euros  de taxes et obtiendrait un gain budgétaire (économie sur les dépenses publiques + gains des taxes) de 2.2 milliards d’euros.
Ce scénario a l’avantage de pouvoir supprimer rapidement les réseaux clandestins et de générer de fortes recettes. Cependant, le nombre de consommateurs ainsi que la consommation augmentant, ce scénario entraînerait des dommages sanitaires massifs. La \textbf{table \ref{tab:synthese3}} synthétise ces résultats.
 

\begin{table}[p]\centering
 \begin{tabular}{p{3.8cm}|cccccc}
  &Statu quo& Légalisation		& Variation  \\
  &   		& dans un cadre		&						\\
  &  		& concurrentiel	&			 			\\\hline \hline   
  Usagers quotidiens (en milliers)  & 550  	&943  	&+71.5 \%  	 \\\hline
  Volume trafic  (en tonnes)  		& 277  	&544  	&+96.4 \%  	 \\\hline
  Prix de vente (en euros)  		& 6  	&5.4&-10.0 \%  \%\\\hline
  Coût d’acquisition (en euros) 	& 8.4	&5.4 		&-35.7 \%  \\\hline
  Dépenses publiques (en millions d’euros) 	& 568  	&76       &-86.4 \%  \\ \hline
  Recettes publiques (en millions d’euros)	&0		& 1764	&\\\hline
 \end{tabular}
 \label{tab:synthese3}
 \caption{Tableau de synthèse du scénario 3 \cite{terraNova_rapport}}
\end{table} 
 
 
\section{Synthèse}

	Le scénario le plus adapté est donc le scénario 2 qui est le seul à permettre une régulation du nombre de consommateurs de cannabis, tout en générant des profits pour l'Etat ainsi qu'un gisement d'emplois.

  			\chapter{Technologies}

    \section{Tincan}
    \section{Greylog}
    \section{Autres}
  			\chapter*{Conclusion}
\addcontentsline{toc}{chapter}{Conclusion  \guy{Pierre}}

Dans le cadre de notre projet de quatrième année en informatique à l'INSA, nous devons mettre en place une étude de trace sur le business game E-Yaka. Le but de cette opération est de collecter des données sur le comportement des apprenants, ce qui permettra d'améliorer l'application, aussi bien sur le plan ergonomique que pédagogique, et de transmettre des informations utiles à l'encadrant de la partie (par exemple, le fait qu'un apprenant ne se connecte plus et a visiblement abandonné la formation).


Nous avons d'abord étudié le jeu E-Yaka, qui met en place une simulation de gestion d'entreprise assez poussée. L'application est immersive : les apprenants jouent le rôle de véritables employés de l'entreprise, avec la possibilité de démarcher personnellement des clients. Cet aspect est sans doute l'un des points forts de l'application, car il permet un travail d'équipe tout en offrant aux apprenants la possibilité de se démarquer, ce qui aura un impact positif sur leur motivation.


Nous nous sommes ensuite renseignés sur les Learning Analytics, une discipline assez récente qui consiste à étudier des données en rapport avec une formation, pour comprendre le processus d'apprentissage et l'améliorer. Malgré des problèmes éthiques qui peuvent surgir dans certaines circonstances, les Learning Analytics ont un fort potentiel pour le futur de l'éducation, notamment pour créer des formations qui s'adaptent aux besoins et au fonctionnement des différents profils d'élèves.


Pour la mise en pratique du projet, nous avons fait des recherches sur les outils existants pour mettre en place des logs. Le plus prometteur semble être Experience API, une norme moderne conçue spécifiquement pour stocker des données en rapport avec l'apprentissage. De plus, une bibliothèque pour simplifier l'utilisation d'Experience API existe en PHP, le langage dans lequel E-Yaka est programmé. Il s'agit donc probablement de la technologie que nous allons retenir.


La prochaine étape du projet consiste à établir une spécification fonctionnelle pour le travail à faire. Pour cela, nous devrons décider précisément à quelles questions nous voulons être en mesure de répondre avec l'étude de traces, ce qui nous permettra de préciser le cahier des charges, et par la suite, de choisir des solutions d'implémentation de façon définitive.
		\cleardoublepage 
   %     \nocite{*}
  %      \bibliographystyle{acm}
      \printbibliography
      
      
      \pagestyle{empty}
      \cleardoublepage~
      \newpage
      ~\vspace{5cm}
      \thispagestyle{empty}
      
      \begin{center}
      	\textbf{Abstract}
      \end{center}
Should France change its stance about recreational marijuana? The French government strictly enforces the repression of marijuana production, distribution, and consumption. This policy has a visible impact on society: on top of its prohibitive costs, social issues, such as criminality or prison overcrowding, may be consequences of the policy. We investigated to what extent these problems really are caused by the repressive governmental stance, and whether or not the country could benefit from a change of policy.

Many states already have more liberal legal frameworks for the use of marijuana. We used as our primary example that of the American state of Colorado, in which the consumption, growing and buying of marijuana has been legal for all citizens over 21 since 2014. We noted that the economic and social consequences of legalisation there were globally positive, considering for instance that criminality had dropped and some 10.000 to 18.000 jobs had been created. (related to marijuana?)

We also compared the current repression policy to the one enforced in the 1920s in the USA on alcohol. Indeed, the situation back then bears much resemblance to the current one: like alcohol, cannabis may induce dependence and cause health problems (which in the case of cannabis, is currently disputed by the scientific community). Moreover, prohibition makes ground in both cases for organised traffic and gang violence.
It appears that the legalisation on alcohol in the 1930s and on cannabis in Colorado, has greatly reduced criminality. Moreover, states have profited from legalisation, by being able to regulate the market and perceive taxes. However, these comparisons have made obvious the difference between depenalisation and legalisation, in a way we shall further explain in the full paper.


        
\end{document}